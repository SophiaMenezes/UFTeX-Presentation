\documentclass{beamer}
\usepackage{amsfonts,amsmath,oldgerm}
\usetheme{uftex}


\newcommand{\testcolor}[1]{\colorbox{#1}{\textcolor{#1}{test}}~\texttt{#1}}

\usefonttheme[onlymath]{serif}

\titlebackground*{assets/CapaNova1.png}

\newcommand{\hrefcol}[2]{\textcolor{cyan}{\href{#1}{#2}}}

\title{Universidade Federal do Tocantins}
\subtitle{Usando \LaTeX\ para Apresentação de Estágio}
\course{Ciência da Computação}
\author{Sophia Menezes Pontes}
%\date{2024}

\begin{document}
\maketitle

\section{Introdução}

\begin{frame}{Beamer para Apresentação Final - Estágio Obrigatório}
\begin{itemize}
\item Beamer é uma das classes de documentos mais populares e poderosas para apresentações em \LaTeX
\item O Beamer possui um
\hrefcol{http://www.ctan.org/tex-archive/macros/latex/contrib/beamer/doc/beameruserguide.pdf}{manual do usuário}
detalhado

\end{itemize}
\end{frame}

\begin{frame}{Beamer vs. PowerPoint}
Comparado ao PowerPoint, usar \LaTeX\ é melhor porque:
\begin{itemize}
\item Você escreve o conteúdo, o computador faz a formatação!
\item Produz um \texttt{pdf}: sem problemas com fontes, fórmulas,
versões de programas
\item Mais fácil de manter estilo, fontes, destaques, etc., consistentes
\item A formatação de matemática em \TeX\ é a melhor:
\begin{equation*}
\mathrm{i},\hslash\frac{\partial}{\partial t} \Psi(\mathbf{r},t) =
-\frac{\hslash^2}{2,m}\nabla^2\Psi(\mathbf{r},t)

V(\mathbf{r})\Psi(\mathbf{r},t)
\end{equation*}
\end{itemize}
\end{frame}

\begin{frame}[fragile]{Como começar?}
Para começar a trabalhar com o \texttt{uftex - Presentation}, inicie um documento \LaTeX\ com o preâmbulo:
\begin{block}{Comandos para iniciar o Documento}
\verb|\documentclass{beamer}|\\
\verb|\usetheme{uftex}|\\
\verb|\begin{document}|\\
\verb|\begin{frame}{Hello, world!}|\\
\verb|\end{frame}|\\
\verb|\end{document}|\\
\end{block}
\end{frame}

\begin{frame}[fragile]{Página de Título}
Para definir uma página de título, você chama alguns comandos:
\begin{block}{Os Comandos para a Página de Título}
\begin{verbatim}
\title{Título Exemplo}
\subtitle{Subtítulo Exemplo}
\author{Primeiro Autor, Segundo Autor}
\date{\today}
\end{verbatim}
\end{block}
Você pode então escrever a página de título com \verb|\maketitle|.

Para definir uma \textbf{imagem de fundo}, use o comando \verb|\titlebackground|
antes de \verb|\maketitle|; seu único argumento é o nome (ou caminho) de um arquivo gráfico.

Se você usar a \textbf{versão com asterisco} \verb|\titlebackground*|, a imagem
será cortada para uma visão dividida no lado direito do slide de título.

\end{frame}

\begin{frame}[fragile]{Escrevendo um Slide Simples}
\framesubtitle{É realmente fácil!}
\begin{itemize}[<+->]
\item Um slide típico tem listas com marcadores
\item Estes podem ser revelados em sequência
\end{itemize}
\begin{block}{Código para uma Página com uma Lista de Itens}
\begin{verbatim}
\begin{frame}{Escrevendo um Slide Simples}
\framesubtitle{É realmente fácil!}
\begin{itemize}[<+->]
\item Um slide típico tem listas com marcadores
\item Estes podem ser revelados em sequência
\end{itemize}\end{frame}
\end{verbatim}
\end{block}
\end{frame}

\section{Personalização}

\footlinecolor{uftexyellow}
\begin{frame}[fragile]{Mudando o Estilo do Slide}
\begin{itemize}
\item Você pode selecionar o estilo de slide branco ou \textit{maincolor} \textbf{no
préâmbulo} com \verb|\themecolor{white}| (padrão) ou \verb|\themecolor{main}|
\begin{itemize}
\item Você \emph{não} deve mudar isso dentro do documento: Beamer não gosta
\item Se você \emph{realmente} precisar, pode ser necessário adicionar
\verb|\usebeamercolor[fg]{normal text}| no slide
\end{itemize}
\item Você pode mudar a \textbf{cor da linha de rodapé} com
\verb|\footlinecolor{color}|
\begin{itemize}
\item Coloque o comando \emph{antes} de um novo \verb|frame|
\item Há quatro cores pré-definidas para isso:
\testcolor{maincolor}, \testcolor{uftexyellow},
\testcolor{uftexgreen}, \testcolor{uftexdarkgreen}
\item O padrão é sem linha de rodapé; você pode usá-lo com
\verb|\footlinecolor{}|
\end{itemize}
\end{itemize}
\end{frame}

\begin{frame}[fragile]{Blocos}
\begin{columns}
\begin{column}{0.3\textwidth}
\begin{block}{Blocos Padrão}
Estes têm uma cor coordenada com a linha de rodapé (e cinza no tema azul)
\begin{verbatim}
\begin{block}{título}
conteúdo...
\end{block}
\end{verbatim}
\end{block}
\end{column}
\begin{column}{0.7\textwidth}
\begin{colorblock}[black]{uftexlightgreen}{Blocos de Cor}
Semelhante aos da esquerda, mas você escolhe a cor. O texto será branco por
padrão, mas você pode defini-lo com um argumento opcional.
\small
\begin{verbatim}
\begin{colorblock}[black]{uftexlightgreen}{título}
conteúdo...
\end{colorblock}
\end{verbatim}
\end{colorblock}
As cores pré-definidas para os blocos de cor são: \testcolor{uftexlightgreen}, , \testcolor{uftexlilla},
\testcolor{maincolor}, \testcolor{uftexdarkgreen}, e
\testcolor{uftexyellow}.
\end{column}
\end{columns}
\end{frame}

\footlinecolor{}
\begin{frame}[fragile]{Usando Cores}
\begin{itemize}
  \item Você pode usar cores com o comando \verb|\textcolor{<nome da cor>}{texto}|
  \item As cores são definidas no pacote \texttt{uftexcolor}:
  \begin{itemize}
  \item Cores principais: \testcolor{maincolor} e \testcolor{uftexgrey}
  \item Três tons de verde: \testcolor{uftexlightgreen}, \testcolor{uftexgreen}, \testcolor{uftexdarkgreen}
  \item Cores adicionais: \testcolor{uftexyellow}, \testcolor{uftexred}, \testcolor{uftexlilla}
  \end{itemize}
  \item Use \verb|\alert{}| para trazer o foco em algum lugar.
  \item Como por \alert{exemplo!}
\end{itemize}
\end{frame}


\begin{frame}[fragile]{Adicionando imagens}
\begin{columns}
\begin{column}{0.7\textwidth}
Como adicionar imagens no \LaTeX\ Beamer:
\begin{block}{Código para Adicionar Imagens}
\begin{verbatim}
\usepackage{graphicx}
% ...
\includegraphics[width=\textwidth]
{assets/exemplo_Ciencia-da-computacao}
\end{verbatim}
\end{block}
\end{column}
\begin{column}{0.35\textwidth}
\includegraphics[width=\textwidth]
{assets/exemplo_Ciencia-da-computacao}
\end{column}
\end{columns}
\end{frame}

\begin{frame}[fragile]{Dividindo em Colunas}
Dividir a página é fácil!
\begin{columns}
\begin{column}{0.6\textwidth}
Esta é a primeira coluna
\end{column}
\begin{column}{0.3\textwidth}
E esta é a segunda
\end{column}
\end{columns}
\begin{block}{Código de Coluna}
\begin{verbatim}
\begin{columns}
    \begin{column}{0.6\textwidth}
        Esta é a primeira coluna
    \end{column}
    \begin{column}{0.3\textwidth}
        E esta é a segunda
    \end{column}
    % Pode haver mais!
\end{columns}
\end{verbatim}
\end{block}
\end{frame}

\begin{chapter}[assets/background (1)]{maincolor}{Slides com Tópicos a serem abordados}
\begin{itemize}
\item Slides de capítulo
\item Slides com imagem lateral
\end{itemize}
\end{chapter}

\footlinecolor{uftexred}
\begin{frame}[fragile]{Slides de Capítulo}
\begin{itemize}
\item Iniciados com \verb|\begin{chapter}[<imagem>]{<cor>}{<título>}|
\item Há sete cores pré-definidas: \testcolor{maincolor}, 
\testcolor{uftexdarkgreen}, \testcolor{uftexfgreen}, 
\testcolor{uftexlightgreen}, \testcolor{uftexred}, \testcolor{uftexyellow}, 
\testcolor{uftexlilla}.

\end{itemize}
\end{frame}

\begin{sidepic}{assets/ExemploFormatura}{Slides com Imagem Lateral}
\begin{itemize}
\item Iniciado com \texttt{$\backslash$begin\{sidepic\}\{<imagem>\}\{<título>\}}
\end{itemize}
\end{sidepic}

\footlinecolor{maincolor}
\begin{frame}
\frametitle{Fontes}
\begin{itemize}
\item A principal tarefa das fontes é ser legível
\item Existem boas fontes...
  \begin{itemize}
  \item {\textrm{Fontes serifadas}}
  \item {\textsf{Fontes sem serifa}}
  \end{itemize}
\item ... e fontes não tão boas:
  \begin{itemize}
  \item {\texttt{Nunca use monoespaço para texto normal}}
  \item {\frakfamily Fontes góticas ou caligráficas: devem sempre ser evitadas}
\end{itemize}
\end{itemize}
\end{frame}

\begin{frame}[fragile]{Slides Finais}
\begin{itemize}
\item Para inserir um slide final com o título ou agradecimentos finais, use \verb|\backmatter|.
      \begin{itemize}
      \item O título também aparece nos rodapés junto com o nome do autor, você pode alterar este texto com \verb|\footlinepayoff|
      \item Você pode remover o título do slide final com \verb|\backmatter[notitle]|
      \end{itemize}
\end{itemize}
\end{frame}

\section{Resumo}

\begin{frame}
\frametitle{Boa Sorte!}
\begin{itemize}
\item O suficiente para uma introdução!
\end{itemize}
\end{frame}

\begin{frame}

Este modelo é baseado em \hrefcol{https://www.overleaf.com/latex/templates/sintef-presentation/jhbhdffczpnx}{SINTEF Presentation} de \hrefcol{mailto
.zenith@sintef.no}{Federico Zenith}.

\vspace{\baselineskip}



\end{frame}

\backmatter
\end{document}
